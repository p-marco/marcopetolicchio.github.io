%------------------------------------
% Dario Taraborelli
%
% url: http://nitens.org/cv/cv.pdf
%------------------------------------


%!TEX TS-program = xelatex
%!TEX encoding = UTF-8 Unicode

\documentclass[10pt, a4paper]{article}
\usepackage{fontspec} 
\usepackage{verbatim,placeins} 

% DOCUMENT LAYOUT
\usepackage{geometry} 
\geometry{a4paper, textwidth=5.6in, textheight=9.5in, marginparsep=7pt, marginparwidth=.6in}
%\geometry{a4paper, textwidth=5.2in, textheight=9.5in, marginparsep=6em, marginparwidth=6em}
\setlength\parindent{0in}
\setlength{\parskip}{0in}

% FONTS
\usepackage[usenames,dvipsnames]{color}
\usepackage{xunicode}
\usepackage{xltxtra}
\defaultfontfeatures{Mapping=tex-text}
\setromanfont [Ligatures={Common}, Numbers={OldStyle}]{EB Garamond}
%\setmainfont [Ligatures={Common}, Numbers={OldStyle}, Scale=0.9]{PT Sans}
%\setromanfont[Ligatures={Common}, SmallCapsFont={OfficinaSanITC-BookSC}, BoldFont={OfficinaSanITC-MediumOS}, ItalicFont={Officina Serif ITC Book Italic OS}]{Officina Serif ITC Book}
%\setmonofont[Scale=0.7]{Monaco}
\setmonofont[Scale=0.7]{IBM Plex Mono}
%\setsansfont [Ligatures={Common}, SmallCapsFont={OfficinaSanITC-BookSC}, BoldFont={OfficinaSanITC-MediumOS}, ItalicFont={Officina Sans ITC Book Italic OS}]{Officina Sans ITC Book Italic OS} 

% ---- CUSTOM COMMANDS
\newcommand{\amper}{{\fontspec[Scale=.95]{Adobe Caslon Pro}\selectfont\itshape\&}}
%\newcommand{\amper}{{\fontspec[Scale=.95]{Linux Libertine O}\selectfont}}
\newcommand{\html}[1]{\href{#1}{\scriptsize\textsc{[html]}}}
\newcommand{\pdf}[1]{\href{#1}{\scriptsize\textsc{[pdf]}}}
\newcommand{\thesispdf}[1]{\href{#1}{\scriptsize\textsc{[thesis pdf]}}}
\newcommand{\doi}[1]{\href{http://doi.org/#1}{\scriptsize\textsc{[doi: #1]}}}
\newcommand{\bib}[1]{\href{#1}{\scriptsize\textsc{[bib]}}}
\newcommand{\isbn}[1]{{\scriptsize\textsc{[isbn: #1]}}}
\newcommand{\issn}[1]{{\scriptsize\textsc{[issn: #1]}}}
\newcommand{\homepage}[1]{\href{http://#1}{\scriptsize\textsc{[url: #1]}}}
%\usepackage{placeins}

\usepackage{marginnote}
\newcommand{\years}[1]{{\footnotesize\marginnote{#1}}}
\renewcommand*{\raggedleftmarginnote}{}
\setlength{\marginparsep}{7pt}
\reversemarginpar

% HEADINGS
\usepackage{sectsty} 
\usepackage[normalem]{ulem} 
\sectionfont{\mdseries\upshape\Large}
\subsectionfont{\mdseries\scshape\normalsize} 
\subsubsectionfont{\mdseries\upshape\large} 

\usepackage{multicol}

% PDF SETUP
% ---- FILL IN HERE THE DOC TITLE AND AUTHOR
\usepackage[xetex, bookmarks, colorlinks, breaklinks, pdftitle={Marco Petolicchio - CV},pdfauthor={Marco Petolicchio}]{hyperref}  
\hypersetup{linkcolor=blue,citecolor=blue,filecolor=black,urlcolor=MidnightBlue} 

\begin{document}
\reversemarginpar
{{\huge Marco Petolicchio}\\[1.2cm]
\parbox{.5\linewidth}{

Křížkovského 10\\
771 80 Olomouc, Czech Republic \\[.4cm]
• \href{tel:+393289594722}{+39 328 95 94 722}\\
%\textsc{f}\quad\texttt{+1 415-882-0495}\\
%\href{http://marcopetolicchio.com}{marcopetolicchio.com}\\
• \href{mailto:marco.petolicchio@gmail.com}{marco.petolicchio@gmail.com}\\
• \href{mailto:marco.petolicchio01@upol.cz}{marco.petolicchio01@upol.cz}\\
}
\parbox{.5\linewidth}{
• \href{http://marcopetolicchio.com}{marcopetolicchio.com}\\
• \href{http://www.linkedin.com/in/marcopetolicchio/}{linkedin.com/in/marcopetolicchio}\\
• \href{http://upol.academia.edu/MarcoPetolicchio}{upol.academia.edu/MarcoPetolicchio}\\
• \href{https://www.researchgate.net/profile/Marco_Petolicchio}{researchgate.net/profile/Marco\_Petolicchio}\\
%{\fontspec{Entypo Social}{}}\hspace{.26cm}\href{http://twitter.com/readermeter}{twitter.com/readermeter}\\
• \href{http://orcid.org/0000-0001-7017-7862}{orcid.org/0000-0001-7017-7862}\\
%{\fontspec{Entypo Social}{}}\hspace{.26cm}\href{https://scholar.google.com/citations?user=4LS5_58AAAAJ}{scholar.google.com}\\%
• \href{http://github.com/p-marco/}{github.com/p-marco}\\
%$\vcenter{\hbox{\includegraphics[scale=0.06]{orcid.png}}}$
}\\[.4cm]



\paragraph{Areas of specialization} \hfill \\%Philosophy of psychology\\[.2cm]
Computational linguistics • Digital humanities • Language acquisition • Syntax theory 

%\tableofcontents



\section{Research Experience}
\noindent
\years{2017-present}\textbf{PhD student in \textsc{Italian Linguistics}}\\
Dept. of Romance Studies, Palacky University Olomouc • {\textsc{advisor:} Dr. Francesco Bianco}



%\hrule
\section{Education}
\noindent

\years{2014}\textbf{Master of Arts in \textsc{General Linguistics}} (\textit{with honors}) \thesispdf{http://www.academia.edu/9213365/Ergativit\%C3\%A0_in_ittito._Un_approccio_minimalista_-_Tesi_di_Laurea_Magistrale} \\
Sapienza Università di Roma •
{ \textsc{advisors:} Prof. Paolo Di Giovine; Prof. Marianna Pozza}\\

\years{2012}\textbf{Bachelor of Arts in \textsc{Romance Philology}} \thesispdf{http://www.academia.edu/31026006/Uno_studio_sulle_Petrose_di_Dante_-_Tesi_di_Laurea_Triennale}\\
Sapienza Università di Roma • \textsc{advisor:} Prof. Stefano Asperti}



\section{Teaching Experience}
\noindent
\textbf{Assistant Lecturer for undergraduate courses}\\ 
Dept. of Romance Studies, Palacky University Olomouc
\begin{itemize}
\item 2018 \textbf{Syntax italštiny 2}
\item 2017 \textbf{Jazykový seminář italštiny 4}
\end{itemize}

\years{2017-2018}\textbf{Italian Lecturer}\\
Transperfect Language School, Na Střelnici 342/46, 779 00 Olomouc


\section{Academic Services}



\subsection{Projects}
\noindent\years{2017-present}{\textsc {\textbf{Czech-IT!} • A linguistic corpus of Czech learners acquiring the italian language}} \\ Co-founder: Digital strategist and Computational architecture \\
\homepage{czech-it.github.io} \doi{10.5281/zenodo.824985} \\

\noindent\years{2016-present}{\textsc{\textbf{PoLet500} • Polemiche Letterarie del Cinquecento}} \\ IT Supervisor: Web development and programming, Workflow \\ 
\textsc{Patronage:} Università degli Studi del Molise; \textsc{Partner:} Filozofski fakultet u Splitu\\ 
\homepage{nuovorinascimento.org/polet500} \isbn{978-88-942416-8-6} \doi{10.5281/zenodo.882998} 



\subsection{Editor and reviewer}

\noindent\years{2017-present}\textsc{\textbf{Editor}, GLoDIUm - Glossario di Informatica Umanistica} \isbn{978-88-942416-9-3}


\noindent\years{2017-present}\textsc{\textbf{Editor}, FRI - Filologia Risorse Informatiche} \issn{2496-6223} 


\subsection{Organization of scientific events}


\noindent\years{2019}\textsc{\textbf{CIFRE 2}}. April 11--13, 2019, Univerzita Palackého v Olomouci.\\ 

\noindent\years{2017}\textsc{\textbf{Loquit 2017}}. November 2--3, 2017, Univerzita Palackého v Olomouci. \homepage{loquit.github.io}  







\section{Publications and talks}




\subsection{Refereed conference papers}

M. Bolpagni, \textbf{M. Petolicchio}. Introduzione al corpus linguistico Czech-IT! • 7° Convegno Annuale AIUCD 2018 • Book of Abstracts. Patrimoni culturali nell'era digitale. Memorie, culture umanistiche e tecnologie. Università degli Studi di Bari. January 31\textsuperscript{st}-February 3\textsuperscript{rd}, 2018. \textsc{pp:} 74-77 . \isbn{forthcoming}.





\subsection{Invited speaker}


M. Bolpagni, \textbf{M. Petolicchio}. Presentazione del corpus linguistico Czech-IT a Nitra (SK) •  Univerzita Konštantína Filozofa v Nitre. April 26\textsuperscript{th}, 2018. \pdf{https://www.researchgate.net/publication/327222171_Presentazione_del_corpus_linguistico_Czech-IT_a_Nitra_SK} \\

\textbf{M. Petolicchio}, M. Bolpagni. Introduzione a Czech-IT! e applicazioni computazionali alla corpus linguistics • Università degli Studi di Udine. December 1\textsuperscript{st}, 2017. \pdf{http://www.academia.edu/35315516/_UniUD_1_12_2017_Introduzione_a_Czech-IT_e_applicazioni_computazionali_alla_corpus_linguistics} \\

\textbf{M. Petolicchio}. Introducing Czech-IT!: a data-based approach for the study of the Second Language Acquisition • Dept. of General Linguistics, Palacký University Olomouc. November 28\textsuperscript{th}, 2017. \pdf{http://www.academia.edu/35278761/Introducing_Czech-IT_a_data-based_approach_for_the_study_of_the_Second_Language_Acquisition} 


\subsection{Conference talks}



\textbf{M. Petolicchio}, M. Bolpagni. Introduzione a Czech-IT! e applicazioni computazionali alla corpus linguistics • Loquit 2017, Palacký University Olomouc. November 3\textsuperscript{rd}, 2017. \pdf{http://www.academia.edu/35256167/Introduzione_a_Czech-IT_e_applicazioni_computazionali_alla_corpus_linguistics} \\

\textbf{M. Petolicchio}. Piccole note di dialettologia dantesca • Loquit 2017, Palacký University Olomouc. November 2\textsuperscript{nd}, 2017. \pdf{http://www.academia.edu/35256139/Piccole_note_di_dialettologia_dantesca} \\

F. Bianco, B. Pace, \textbf{M. Petolicchio}. Portali e riviste letterarie: il progetto NodIT • Riviste letterarie 2.0, Pole multimédia (Campus Schuman), Aix-en-Provence. October 5\textsuperscript{th}, 2017. 


%\subsection{3. Edited volumes}

%\subsection{5. Refereed conference posters}





\subsection{Press}

L.Forti, \textbf{M. Petolicchio}, G. Spinelli, \emph{Linguaggi e potere}, Interview to Noam Chomsky, Roma Città del Vaticano. March 03\textsuperscript{rd}, 2014. \html{http://www.letterefilosofia.com/linguaggi-e-potere-intervista-a-noam-chomsky/} 



\subsection{Other publications}

\textbf{M. Petolicchio}. Risorse digitali per gli studi linguistici: un breve appunto, in \textit{Filologia Risorse Informatiche}. January-February, 2017. \issn{2496-6223} \pdf{http://www.academia.edu/31229297/Risorse_digitali_per_gli_studi_linguistici_un_breve_appunto_in_Filologia_Risorse_Informatiche_gennaio-febbraio_2017} \html{https://fri.hypotheses.org/658} \\

A. F. Caterino, \textbf{M. Petolicchio}. Polemiche Letterarie del Cinquecento (PoLet500): presentazione del progetto, in \textit{Filologia Risorse Informatiche}. January-February, 2017. \issn{2496-6223} \pdf{http://www.academia.edu/30975804/Polemiche_Letterarie_del_Cinquecento_PoLet500_presentazione_del_progetto_in_Filologia_Risorse_Informatiche_gennaio-febbraio_2017} \html{https://fri.hypotheses.org/651} 

%\subsection{7. White papers \amper{} reports}


%\subsection{8. Data}

%\section{Press}


\vfill
%\hrulefill
\begin{center}
{\scriptsize  Last updated: \today\- • \href{http://marcopetolicchio.com}{marcopetolicchio.com}}
\end{center}

\end{document}